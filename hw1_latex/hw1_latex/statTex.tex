\section{ex1.2a}
\subsection{Question 1}
Expected value is defined as:
$$E(f) = \sum_{\omega \in \Omega} f(\omega)P(\omega)$$
Variance is defined as:
$$var(f) = \sum_{\omega \in \Omega} (f(\omega)E(f))^2P(\omega)$$

\subsection{Question 2}
\begin{itemize}
	\item \textbf{Dice A: } \newline
	$$
		E(A) = \frac{1}{6} (4+4+2+4+1+1 ) = \frac{16}{6}
	$$
		
	$$
		var(A) = \frac{3}{5}(4-\frac{16}{6})^2 + \frac{1}{5}(2-\frac{16}{6})^2 + \frac{2}{5}(1-\frac{16}{6})^2 = \frac{34}{15}
	$$

	\item \textbf{Dice B: } \newline
	$$
		E(A) = \frac{1}{6} (3+6+3+3+4+3) = \frac{11}{3}
	$$
		
	$$
		var(A) = \frac{4}{5}(3-\frac{11}{3})^2 + 
			\frac{1}{5}(6-\frac{11}{3})^2 + 
			\frac{1}{5}(4-\frac{11}{3})^2 = \frac{22}{15}
	$$

	\item \textbf{Dice C: } \newline
	$$
		E(A) = \frac{1}{6} (5+5+2+1+1+1) = \frac{5}{2}
	$$
		
	$$
		var(A) = \frac{2}{5}(5-\frac{5}{2})^2 + 
			\frac{1}{5}(2-\frac{5}{2})^2 + 
			\frac{3}{5}(1-\frac{5}{2})^2 = \frac{39}{10}
	$$
\end{itemize}

\subsection{Question 3}
	@TODO

\section{ex 1.2b}
	
\subsection{Question 1}
Let $C$ be a Bernoulli random variable that identify if a person has or not cold. \\
Let $B$ be a Bernoulli random variable that identify if a person has or not backpain. 
\subsection{Question 2}
The domain of bot $C$ and $B$ is $\{0, 1\}$ since they are Bernoulli.
\subsection{Question 3}
\begin{itemize}
	\item a) $ P( B=1 | C=1 ) = 25\% $
	\item b) $ P( C=1 ) = 4\% $
	\item c) $ P( B=1 | C=0 ) = 15\% $
\end{itemize}
\subsection{Question 4}
We want to obtain the following probability: $ P( C=1 | B=1 ) $ \\
Applying Bayes theorem:
$$
	P( C=1 | B=1 ) = \frac{P( B=1 | C=1 ) P( C=1 )}{P( B=1 )}
$$
Both $ P( B=1 | C=1 ) $ and $ P( C=1 ) $ are known. \\
Now we need to calculate only $P(B=1)$:
$$
	P(B=1) = P( B=1 | C=1 ) \cdot P( C=1 ) + P( B=1 | C=0 ) \cdot P( C=0 ) =
$$
$$
	= \frac{25}{100} \cdot \frac{4}{100} + \frac{15}{100} \cdot \frac{96}{100} = 0.154
$$
Hence:
$$
	P( C=1 | B=1 ) = \frac{P( B=1 | C=1 ) P( C=1 )}{P( B=1 )} = \frac{0.25 \cdot 0.04}{0.154} = \frac{5}{77}
$$


\section{ex 1.2c}
	
\subsection{Question 1}
Since the random variables $A$, $B$, $C$, $D$ are i.i.d. the probability of the disjunction is obtained simply by multiplying the single probabilities, hence:
$$
	P(A=1 \cap B=1 \cap C=1 \cap D=1) = P(A=1) \cdot P(B=1) \cdot P(C=1) \cdot P(D=1) = (\frac{2}{3})^4
$$

\subsection{Question 2}
$$
	P(A=0 \cap B,C,D = 1) = \frac{1}{3} \cdot (\frac{2}{3})^3
$$
\subsection{Question 3}
\begin{lstlisting}
TRIES = 10000000
PROB = 2. / 3
a = np.random.binomial(1, PROB, TRIES)
b = np.random.binomial(1, PROB, TRIES)
c = np.random.binomial(1, PROB, TRIES)
d = np.random.binomial(1, PROB, TRIES)
success = 0
for i in range(TRIES):
    if a[i] == 0 and b[i] == 1 and c[i] == 1 and d[i] == 1:
        success += 1

print (success / TRIES)
\end{lstlisting}

The analytical result is $8/81 \approx 0.09876543209$\\
The result of the simple simulation above (in one specific computation) was $0.0986829$, and they match up to the fourth decimal digit.