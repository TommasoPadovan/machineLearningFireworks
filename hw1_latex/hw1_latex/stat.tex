\newif\ifvimbug
\vimbugfalse

\ifvimbug
\begin{document}
\fi

\exercise{Statistics Refresher}
 

\begin{questions}

%----------------------------------------------

\begin{question}{Expectation and Variance}{8}
Let $\Omega$ be a finite set and $P:\Omega\rightarrow\R$ a probability measure that (by definition) satisfies $P(\omega)\geq0$ for all $\omega\in\Omega$ and $\sum_{\omega\in\Omega}P(\omega)=1$. 
Let $f:\Omega\rightarrow\R$ be an arbitrary function on $\Omega$.

\textbf{1)} Write the definition of expectation and variance of $f$ and discuss if they are linear operators.

\textbf{2)} You are given a set of three dices $\{A,B,C\}$.
The following table describes the outcome of six rollouts for these dices, where each column shows the outcome of the respective dice. 
(Note: assume the dices are standard six-sided dices with values between 1-6)
\begin{equation*}
\begin{array}{r|cccccc}
    A & 4 & 4 & 2 & 4 & 1 & 1 \\
    \hline
    B & 3 & 6 & 3 & 3 & 4 & 3 \\
    \hline
    C & 5 & 5 & 2 & 1 & 1 & 1 
\end{array}
\end{equation*}
Estimate the expectation and the variance for each dice using unbiased estimators. (Show your computations).

\textbf{3)} According to the data, which of them is the ``most rigged''? Why?

\begin{answer}
\textbf{1)} Expected value is defined as:
$$E(f) = \sum_{\omega \in \Omega} f(\omega)P(\omega)$$
Variance is defined as:
$$var(f) = \sum_{\omega \in \Omega} (f(\omega)-E(f))^2P(\omega)$$

$ E[\alpha f(x) + \beta g(x)] = \alpha  E[f(x)] + \beta E[g(x)]$ thanks to the Expectation operator properties. So it is a linear operator. \\
$ Var[\alpha f(x) + \beta g(x)] = \alpha^{2}  Var[f(x)] + \beta^{2} E[g(x)] + 2\alpha \beta Cov[f(x),g(x)] $. So it is not a linear operator \\ \\
\textbf{2)} \begin{itemize}
	\item \textbf{Dice A: } \newline
	$$
	E(A) = \frac{1}{6} (4+4+2+4+1+1 ) = \frac{16}{6}
	$$
	
	$$
	var(A) = \frac{3}{5}(4-\frac{16}{6})^2 + \frac{1}{5}(2-\frac{16}{6})^2 + \frac{2}{5}(1-\frac{16}{6})^2 = \frac{34}{15}
	$$
	
	\item \textbf{Dice B: } \newline
	$$
	E(A) = \frac{1}{6} (3+6+3+3+4+3) = \frac{11}{3}
	$$
	
	$$
	var(A) = \frac{4}{5}(3-\frac{11}{3})^2 + 
	\frac{1}{5}(6-\frac{11}{3})^2 + 
	\frac{1}{5}(4-\frac{11}{3})^2 = \frac{22}{15}
	$$
	
	\item \textbf{Dice C: } \newline
	$$
	E(A) = \frac{1}{6} (5+5+2+1+1+1) = \frac{5}{2}
	$$
	
	$$
	var(A) = \frac{2}{5}(5-\frac{5}{2})^2 + 
	\frac{1}{5}(2-\frac{5}{2})^2 + 
	\frac{3}{5}(1-\frac{5}{2})^2 = \frac{39}{10}
	$$
\end{itemize}

\textbf{3)}
The most rigged dice is the dice with higher observed standard deviation of the result frequencies with respect to the one of a balanced dice.\\
A balanced dice is expected to output any number with probability $1/6$.
\begin{itemize}
	\item \textbf{A} 
	$$
	\sqrt{
		(\frac{2}{6}-\frac{1}{6})^2 +
		(\frac{1}{6}-\frac{1}{6})^2 +
		(\frac{0}{6}-\frac{1}{6})^2 +
		(\frac{3}{6}-\frac{1}{6})^2 +
		(\frac{0}{6}-\frac{1}{6})^2 +
		(\frac{0}{6}-\frac{1}{6})^2
	} = 
	$$
	$$
	= \sqrt{\frac{2}{9}} = \frac{\sqrt{2}}{3} \approx 0.4714
	$$
	\item \textbf{B} 
	$$
	\sqrt{
		(\frac{0}{6}-\frac{1}{6})^2 +
		(\frac{0}{6}-\frac{1}{6})^2 +
		(\frac{4}{6}-\frac{1}{6})^2 +
		(\frac{1}{6}-\frac{1}{6})^2 +
		(\frac{0}{6}-\frac{1}{6})^2 +
		(\frac{1}{6}-\frac{1}{6})^2
	} = 
	$$
	$$
	= \sqrt{\frac{1}{3}} = \frac{1}{\sqrt{3}} \approx 0.5774
	$$
	
	
	\item \textbf{C} 
	$$
	\sqrt{
		(\frac{3}{6}-\frac{1}{6})^2 +
		(\frac{1}{6}-\frac{1}{6})^2 +
		(\frac{0}{6}-\frac{1}{6})^2 +
		(\frac{0}{6}-\frac{1}{6})^2 +
		(\frac{2}{6}-\frac{1}{6})^2 +
		(\frac{0}{6}-\frac{1}{6})^2
	} = 
	$$
	$$
	= \sqrt{\frac{2}{9}} = \frac{\sqrt{2}}{3} \approx 0.4714
	$$
\end{itemize}

Hence the dice $B$ is the most rigged since he has the biggest standard deviation.
\end{answer}

\end{question}

%----------------------------------------------

\begin{question}{It is a Cold World}{7}
Consider the following three statements:
\\
a) A person with a cold has backpain $25\%$ of the time.
\\
b) $4\%$ of the world population has a cold.
\\
c) $15\%$ of those who do not have a cold, still have backpain.

\textbf{1)} Identify random variables from the statements above and define a unique symbol for each of them.\\
\textbf{2)} Define the domain of each random variable.\\
\textbf{3)} Represent the three statements above with your random variables.\\
\textbf{4)} If you suffer from backpain, what are the chances that you suffer from a cold? (Show all the intermediate steps.)

\begin{answer}
	\textbf{1)} Let $C$ be a Bernoulli random variable that identify if a person has or not cold. \\
	Let $B$ be a Bernoulli random variable that identify if a person has or not backpain. \\
	\textbf{2)} The domain of bot $C$ and $B$ is $\{0, 1\}$ since they are Bernoulli.\\
	\textbf{3)} 
	\begin{itemize}
	\item [a)] $ P( B=1 | C=1 ) = 25\% $
	\item [b)] $ P( C=1 ) = 4\% $
	\item [c)] $ P( B=1 | C=0 ) = 15\% $
\end{itemize}
	\textbf{4)} We want to obtain the following probability: $ P( C=1 | B=1 ) $ \\
	Applying Bayes theorem:
	$$
	P( C=1 | B=1 ) = \frac{P( B=1 | C=1 ) P( C=1 )}{P( B=1 )}
	$$
	Both $ P( B=1 | C=1 ) $ and $ P( C=1 ) $ are known. \\
	Now we need to calculate only $P(B=1)$:
	$$
	P(B=1) = P( B=1 | C=1 ) \cdot P( C=1 ) + P( B=1 | C=0 ) \cdot P( C=0 ) =
	$$
	$$
	= \frac{25}{100} \cdot \frac{4}{100} + \frac{15}{100} \cdot \frac{96}{100} = 0.154
	$$
	Hence:
	$$
	P( C=1 | B=1 ) = \frac{P( B=1 | C=1 ) P( C=1 )}{P( B=1 )} = \frac{0.25 \cdot 0.04}{0.154} = \frac{5}{77}
	$$
\end{answer}

\end{question}


%----------------------------------------------

\begin{question}{Journey to THX1138}{10}
	After the success of the \href{http://rosetta.esa.int/}{Rosetta mission}, ESA decided to send a spaceship to rendezvous with the comet THX1138. 
	This spacecraft consists of four independent subsystems $A,B,C,D$. 
	Each subsystem has a probability of failing during the journey equal to $1/3$. 
	\\
	1) What is the probability of the spacecraft $S$ to be in working condition (i.e., all subsystems are operational at the same time) at the rendezvous?
	\\
	2) Given that the spacecraft $S$ is not operating properly, compute	analytically the probability that \textbf{only} subsystem $A$ has failed. 
	\\
	3) Instead of computing the probability analytically, do a simple simulation experiment and compare the result to the previous solution. 
	Include a snippet of your code. 
	\\
	4) An improved spacecraft version has been designed.
	The new spacecraft fails if the critical subsystem $A$ fails, or any two subsystems of the remaining $B,C,D$ fail. 
	What is the probability that \textbf{only} subsystem $A$ has failed, given that the spacecraft $S$ is failing? 
	
	
\begin{answer}
	1)Since the random variables $A$, $B$, $C$, $D$ are i.i.d. the probability of the disjunction is obtained simply by multiplying the single probabilities, hence:
	$$
	P(A=1 \cap B=1 \cap C=1 \cap D=1) = P(A=1) \cdot P(B=1) \cdot P(C=1) \cdot P(D=1) = (\frac{2}{3})^4
	$$\\
	2)$$
	P(A=0 \cap B,C,D = 1 \ | \ S = 0) = \frac{P(S=0 | A= 0 \cap B=1 \cap C=1 \cap D=1) \cdot P(A=1 \cap B=1 \cap C=1 \cap D=1)}{P(S=0)} = 
	$$ \\
	$$
	 = \frac{1 \cdot (\frac{2}{3})^3 \cdot \frac{1}{3}}{1- (\frac{2}{3})^4} = \frac{8}{65}
	$$\\
	3)
\lstinputlisting[language=Python]{rosetta.py}	
%	\begin{lstlisting}
%		TRIES = 10000000
%		PROB = 2. / 3
%		a = np.random.binomial(1, PROB, TRIES)
%		b = np.random.binomial(1, PROB, TRIES)
%		c = np.random.binomial(1, PROB, TRIES)
%		d = np.random.binomial(1, PROB, TRIES)
%		success = 0
%		for i in range(TRIES):
%		if a[i] == 0 and b[i] == 1 and c[i] == 1 and d[i] == 1:
%		success += 1
%		print (success / TRIES)
%	\end{lstlisting}
		
	
	The analytical result is $8/81 \approx 0.09876543209$\\
	The result of the simple simulation above (in one specific computation) was $0.0986829$, and they match up to the fourth decimal digit.

	4) We know that 
	$$
	P(S=0) \ = \ P(S=0 \  | \ A=1) \cdot P(A=1) + P(S=0 \ | \ A=0) \cdot P(A=0) \ = \  ( {{3}\choose{2}} \frac{2}{3} (\frac{1}{3})^2 + (\frac{1}{3})^3) \cdot \frac{2}{3} + 1 \cdot  \frac{1}{3}=\frac{41}{81}
	$$\\
	Hence 
	$$
	P(A=0 \cap B=1 \cap C=1 \cap D=1 \ | \ S=0) \ = $$$$ = \frac{P(S=0 \  | \ A=0 \cap  B=1 \cap C=1 \cap D=1) \cdot P( \ A=0 \cap  B=1 \cap C=1 \cap D=1) }{P(S = 0)}  = 
	$$$$
	= \frac{1 \cdot \frac{1}{3} \cdot (\frac{2}{3})^3}{\frac{41}{81}} = \frac{8}{41}  \approx 0.19512
	$$
\end{answer}
	
\end{question}


\end{questions}
